\documentclass[12pt,a4paper]{article}

\usepackage[utf8]{inputenc}
\usepackage[T1]{fontenc}
\usepackage[spanish]{babel}
\usepackage{enumitem}
\usepackage{hyperref}

\title{Primera Aproximación al Trabajo de Fin de Máster}
\author{Tu Nombre}
\date{\today}

\begin{document}

\maketitle

\section*{Título tentativo}
\textbf{Análisis estático y dinámico de código fuente descargado de internet: detección de software malicioso en repositorios abiertos}

\section*{Descripción general}
Este trabajo de fin de máster tiene como objetivo estudiar técnicas de análisis de malware aplicadas a código fuente obtenido de internet. Se plantea evaluar la presencia de comportamientos maliciosos en repositorios abiertos mediante herramientas y metodologías tanto estáticas como dinámicas, buscando identificar patrones comunes en posibles amenazas encubiertas en proyectos aparentemente legítimos.

\section*{Estructura preliminar y puntos clave}

\subsection*{1. Introducción}
\begin{itemize}[label=--]
    \item Contexto del malware en repositorios públicos.
    \item Motivación del estudio.
    \item Objetivos generales y específicos.
\end{itemize}

\subsection*{2. Estado del arte}
\begin{itemize}[label=--]
    \item Análisis de malware: enfoques estáticos y dinámicos.
    \item Casos documentados de malware en GitHub y otros repositorios.
    \item Herramientas existentes para escaneo de código malicioso.
\end{itemize}

\subsection*{3. Metodología}
\begin{itemize}[label=--]
    \item Criterios para selección de proyectos/repositories.
    \item Entorno controlado para análisis dinámico (sandboxing).
    \item Técnicas de análisis estático (linting, detección de obfuscación, etc.).
    \item Pipeline de análisis automatizado.
\end{itemize}

\subsection*{4. Resultados esperados}
\begin{itemize}[label=--]
    \item Identificación de patrones o firmas maliciosas comunes.
    \item Clasificación de riesgos según tipo de proyecto o lenguaje.
    \item Limitaciones del enfoque propuesto.
\end{itemize}

\subsection*{5. Conclusiones provisionales}
\begin{itemize}[label=--]
    \item Potencial impacto del estudio en la comunidad open-source.
    \item Viabilidad de integrar este análisis en flujos DevSecOps.
    \item Proyecciones para trabajos futuros o ampliaciones del estudio.
\end{itemize}

\end{document}
