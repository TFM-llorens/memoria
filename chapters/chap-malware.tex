\chapter{Malware}

\textit{Contenido: En esta sección se explicará de forma general las partes 
de un malware}
\vspace{1em}

Un malware es un software diseñado para realizar acciones maliciosas en un 
sistema.

En este trabajo, el malware que desarrollaremos tendrá como objetivo encriptar 
todo el disco de la víctima.

El primer desafío consiste en conseguir ejecutar código en el sistema de la 
víctima. Las formas más habituales de lograrlo incluyen: phishing, explotación 
de vulnerabilidades sin parchear o uso de credenciales expuestas. Por ejemplo, 
si se dispone de credenciales que permiten acceder a una VPN, podríamos ejecutar 
código de manera remota.

El segundo desafío es garantizar que el malware no sea bloqueado por un EDR. 
Para ello, desarrollaremos un loader, un componente encargado de preparar el 
entorno y lanzar el malware de manera segura y silenciosa.

\section{Loader}

1. Carga en memoria (In-memory execution)
2. Inyección en procesos legítimos (Process Injection)
3. Evasión de defensas
4. Persistencia
5. Descarga o desempaquetado de payloads

\subsection{DLL Manipulation}

\subsubsection{Phantom DLL Injection}

\subsubsection{DLL reflection}

\input{chapters/malware/sec-process-injection.tex}

\input{chapters/malware/sec-shellcode.tex}
\input{chapters/malware/sec-packer.tex}
