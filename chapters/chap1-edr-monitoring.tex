\chapter{EDRs Monitorización}

\section*{Técnicas de EDR}

\subsection*{Introducción}
Los EDRs (Endpoint Detection and Response) emplean diversas técnicas para
monitorizar y proteger los sistemas contra actividades maliciosas. Estas
técnicas pueden utilizarse de forma individual o en combinación, como el
Desenganche de API y las Llamadas Directas al Sistema.

\subsection*{Técnicas}
\begin{itemize}
    \item \textbf{Uso de APIs no enganchadas}
    \item \textbf{Desenganche en modo usuario}
    \item \textbf{Llamadas indirectas al sistema}
    \item \textbf{Llamadas directas al sistema}
\end{itemize}

\subsection*{Callbacks del Kernel}
\begin{itemize}
    \item Rastreo de Eventos para Windows (ETW)
    \item Interfaz de Escaneo Antimalware (AMSI)
\end{itemize}

\subsection*{Monitorización de Llamadas al Sistema}
La monitorización de llamadas al sistema implica diversas técnicas de enganche:
\begin{itemize}
    \item Enganche de API en línea
    \item Enganche de la Tabla de Direcciones de Importación (IAT)
    \item Enganche de SSDT (Kernel de Windows)
\end{itemize}

La mayoría de los EDRs utilizan la técnica de enganche de API en línea
reemplazando \texttt{mov eax, SSN} con una instrucción de salto incondicional
\texttt{jmp}.
