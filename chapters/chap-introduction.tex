\chapter{Introducción}

\textit{Contenido: Esta sección dará un vistazo general a un ataque típico 
apoyándose en el Cyber Kill Chain de Lockheed Martin y el modelo MITRE ATTACK.}
\vspace{1em}

\section{Reconocimiento}

Recopilando información sobre la víctima

\section{Weaponization}

Se desarrolla un payload malicioso

\section{Entrega}

Como se entrega el payload a la víctima

Vectores típicos: phishing, vulnerabilidades sin parchear, credenciales expuestas.

El objetivo aquí es hacer llegar el malware o el loader al sistema de la víctima.

\section{Explotación}

Como conseguir ejecucion del payload en la víctima

Una vez entregado, se consigue ejecutar código en el sistema.

Puede ser directo (el ransomware se ejecuta) o mediante un loader que se encarga 
de preparar el entorno y luego lanzar el malware.

\section{Instalación}

El malware se instala en el sistema de la víctima.
El loader instala el ransomware, a menudo asegurando persistencia, o simplemente 
prepara el entorno para que el malware se ejecute correctamente.

\section{Command and Control (C2)}

El malware establece comunicación con el atacante para recibir instrucciones.

\section{Acciones sobre el objetivo}

El atacante realiza su objetivo final
- exfiltracion
- encriptacion
- criptominero