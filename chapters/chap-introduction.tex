\chapter{Introducción}

\section{Estructura del trabajo}
El presente trabajo se organiza en capítulos, donde cada capítulo analiza en 
profundidad un ciberataque concreto. Para ofrecer una visión completa y 
equilibrada del fenómeno, en cada capítulo se desarrollan de forma paralela 
las dos perspectivas fundamentales:

\begin{itemize}
    \item \textbf{Perspectiva del atacante (Red Team):} se describe el ciclo de 
    vida completo del cibercriminal aplicado al caso estudiado, junto con las 
    técnicas y herramientas empleadas, los obstáculos encontrados y los 
    beneficios o impactos obtenidos por el atacante.
    \item \textbf{Perspectiva del defensor (Blue Team / analistas):} se expone 
    la respuesta de los equipos de seguridad y analistas forenses: 
    identificación y recolección de evidencias, metodologías de investigación, 
    dificultades técnicas encontradas durante la triage y contención, y medidas 
    de mitigación y remediación aplicadas.
\end{itemize}

Cada capítulo finalizará con una sección de \emph{lecciones aprendidas} y 
recomendaciones prácticas que integren ambas perspectivas, resaltando 
indicadores de compromiso (IoC), puntos de mejora en la postura defensiva y 
propuestas de detección y prevención basadas en las tácticas, técnicas y 
procedimientos (TTPs) observados. Esta organización pretende facilitar el 
análisis comparativo entre ofensiva y defensiva y aportar conclusiones útiles 
tanto para equipos de red team como para equipos de respuesta y defensa.
