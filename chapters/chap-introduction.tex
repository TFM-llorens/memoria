\chapter{Introducción}

\textit{Contenido: Esta sección dará un vistazo general al ataque que se va llevar a cabo, 
apoyándose en el Cyber Kill Chain de Lockheed Martin y el modelo MITRE ATTACK.}
\vspace{1em}

\section{Reconocimiento}

\subsection{Objetivo}

Identificar cualquier elemento dentro de la organización víctima que pueda 
conducir a una ejecución remota de código (RCE).

\begin{itemize}
    \item Vulnerabilidades en servicios expuestos
    \begin{itemize}
        \item Buffer overflow
        \item ROP
    \end{itemize}
    \item Credenciales expuestas
    \begin{itemize}
        \item VPN
        \item RDP
    \end{itemize}
\end{itemize}

\subsection{Metodología}

\begin{enumerate}
    \item Enumeración de servicios expuestos asociados a la víctima.
    \item Búsqueda de vulnerabilidades conocidas (CVEs) asociadas a dichos servicios.
    \item Identificación del mayor número posible de usuarios relacionados con la víctima.
    \item Rastreo de posibles credenciales expuestas en fuentes abiertas.
\end{enumerate}

\subsection{Hallazgos}

En nuestro caso, se ha identificado un servidor web expuesto 
\textbf{vulnerable a técnicas de explotación basadas en ROP chain}, 
lo que podría permitir la ejecución de código arbitrario.



\section{Weaponization}

\subsection{Objetivo}

Crear el payload que se utilizará para explotar la vulnerabilidad identificada 
en el paso anterior.

\subsection{Metodología}

Normalmente se utilizar herramientas en este caso los payloads se crearan a mana 
por fines didacticos.

\section{Entrega}

Como se entrega el payload a la víctima

Vectores típicos: phishing, vulnerabilidades sin parchear, credenciales expuestas.

El objetivo aquí es hacer llegar el malware o el loader al sistema de la víctima.

\section{Explotación}

Como conseguir ejecucion del payload en la víctima

Una vez entregado, se consigue ejecutar código en el sistema.

Puede ser directo (el ransomware se ejecuta) o mediante un loader que se encarga 
de preparar el entorno y luego lanzar el malware.

\section{Instalación}

El malware se instala en el sistema de la víctima.
El loader instala el ransomware, a menudo asegurando persistencia, o simplemente 
prepara el entorno para que el malware se ejecute correctamente.

\section{Command and Control (C2)}

El malware establece comunicación con el atacante para recibir instrucciones.

\section{Acciones sobre el objetivo}

El atacante realiza su objetivo final
- exfiltracion
- encriptacion
- criptominero