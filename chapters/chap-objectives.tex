\chapter{Alcance}

\textit{Contenido: Aquí se presenta la idea del trabajo y qué se pretende conseguir.}
\vspace{1em}

El objetivo de este trabajo es desarrollar un escenario didáctico que permita a los
investigadores de malware mejorar sus capacidades de análisis mediante la comprensión
del proceso completo de creación de una malware. El resultado final del proyecto será
un volcado de memoria de un sistema infectado, el cual servirá como base para ejercicios
de análisis forense (blue team).

Para generar dicho recurso, el núcleo del trabajo se centrará en la concepción, diseño y
desarrollo de un malware realista, su ejecución controlada en un sistema víctima, y la
documentación íntegra del proceso ofensivo llevado a cabo por el red team. De este modo,
el participante que reciba el volcado podrá analizar la infección desde el punto de vista
del analista forense, pero también interpretar las acciones del atacante, entendiendo mejor
el cómo y el por qué de cada artefacto encontrado en memoria.

En definitiva, este trabajo busca aproximar al investigador a la mentalidad del creador de
malware, con el fin de reforzar sus habilidades de detección, interpretación y respuesta
ante amenazas reales.



