\section{Portable Executable (PE)}

El formato \textit{Portable Executable} (PE) es el equivalente en Windows al formato ELF en Linux. 
Se utiliza para representar ejecutables, bibliotecas dinámicas (DLL) y otros ficheros ejecutables en dicho sistema operativo.

\subsection{Sections}
Son el equivalente a los segmentos en un ELF.  
Las secciones indican cómo debe ubicarse la memoria dentro del proceso, 
definiendo distintas regiones con protecciones específicas (lectura, escritura, ejecución).

\subsection{Imports}
Contiene la lista de \texttt{DLLs} que son dependencias del ejecutable.

\subsection{Exports}
Si el archivo PE corresponde a una biblioteca compartida y exporta alguna funcionalidad, 
esta información se registra en la sección de exportaciones.  
A diferencia de Linux, donde las funciones suelen exportarse por defecto, 
en Windows deben exportarse explícitamente.

\subsection{Relocations}
Al habilitar \texttt{ASLR} (Address Space Layout Randomization), 
el ejecutable puede necesitar ser reubicado en memoria.  
Las direcciones a modificar en dicho proceso quedan registradas en esta sección.

\subsection{AddressOfEntryPoint}
Campo que indica la dirección de inicio de la ejecución del programa.

\subsection{Subsystem}
Especifica en qué entorno debe ejecutarse el binario:  
si será una aplicación de consola, una GUI, etc.

\subsection{Virtual Address}
Dirección relativa que tendrá la sección al cargarse en memoria.  
Siempre se calcula como:

\begin{equation}
\text{Dirección en memoria} = \text{ImageBase} + \text{Virtual Address}
\end{equation}

\subsection{Raw Data}
Indica la dirección dentro del archivo PE donde se encuentra físicamente la sección en disco.
