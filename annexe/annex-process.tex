\section{Process}

\subsection{EPROCESS}
Estructura que mantiene toda la información de un proceso en el kernel.

\subsection{Threads}
Representan las hebras de ejecución asociadas al proceso.

\subsection{Handles}
Son equivalentes a los \textit{file descriptors} en Linux.

\subsection{Memory}
Gestión del espacio de direcciones y memoria asignada al proceso.

\subsection{Modules}
Lista de módulos y librerías cargadas en el proceso.

\section{Process Creation (Kernel)}

La creación de un proceso en Windows sigue una serie de pasos específicos a nivel
del kernel.

\subsection*{(1) Inicialización del espacio de direcciones}

\begin{itemize}[leftmargin=1.5cm]
    \item \textbf{Modelo en Linux vs. Windows:} En Linux, los nuevos procesos se crean
    mediante la llamada \texttt{fork}, mientras que en Windows los procesos se inicializan
    completamente desde cero.
    \item \textbf{Mapeo de \texttt{KUSER\_SHARED\_DATA}:} Al crear un nuevo proceso, el
    kernel debe mapear memoria. Una de las primeras acciones es mapear la página de 4KB
    denominada \texttt{KUSER\_SHARED\_DATA}, la cual permite intercambiar información entre
    \textit{user mode} y \textit{kernel mode} sin necesidad de \textit{syscalls}. En esta
    página se encuentran datos como el reloj del sistema y las rutas a directorios del
    sistema.
    \item \textbf{Mapeo del ejecutable:} Posteriormente, el kernel carga el ejecutable
    dentro del proceso. Concretamente, carga el código PE en la sección \texttt{.text}.
    \item \textbf{Mapeo de \texttt{ntdll.dll}:} El kernel carga el módulo \texttt{ntdll.dll},
    el cual actúa como intermediario entre el \textit{user-mode} y el kernel. Este módulo
    contiene las funciones necesarias para la comunicación con el kernel y cumple un rol
    similar al \texttt{ld.so} en Linux.
    \item \textbf{Asignación del PEB (Process Environment Block):} El \texttt{PEB} es una
    estructura de datos en \textit{user mode} (aproximadamente 1--2 páginas de memoria) que
    contiene:
    \begin{itemize}
        \item Variables de entorno.
        \item Línea de comandos completa con la que se inició el proceso.
        \item Directorio de trabajo.
        \item Lista de módulos cargados (\texttt{Ldr}).
        \item Punteros al \texttt{stack} y \texttt{heap}.
        \item Dirección base de la imagen.
    \end{itemize}
\end{itemize}

\subsection*{(2) Creación del hilo inicial}

\begin{itemize}[leftmargin=1.5cm]
    \item \textbf{Mapeo del stack:} Se asigna el espacio de pila necesario para el hilo
    principal.
    \item \textbf{Mapeo del TEB (Thread Environment Block):} El \texttt{TEB} es una
    estructura pequeña (2 páginas aprox.) que almacena información específica de cada hilo.
    Su función es equivalente al \texttt{PEB}, pero a nivel de hilo.
    \item \textbf{Inicialización del puntero de ejecución:} Finalmente, el puntero de
    ejecución se coloca en la función \texttt{ntdll.LdrInitializeThunk}, encargada de
    completar la carga dinámica del proceso.
\end{itemize}
