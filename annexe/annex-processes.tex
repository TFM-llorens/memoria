\section{Proceso}

Cada proceso tiene su propio espacio de direcciones virtual,
Dentro de ese espacio un proceso esta dividido en secciones,

\subsection{Secciones de un proceso}

Listado en orden de mayor direcciones de memoria a menor:

\subsubsection{Paginas del Kernel}

Esta seccion ocupa lo mismo en todos los procesos del sistema,
y corresponde a todas las paginas que contienen el kernel del sistema operativo,
Como todas las paginas de kernel de todos los procesos hacen referencia a los mismos frames de memoria fisica,
no se desperdicia espacio cargando el kernel en cada proceso,

Para acceder a la memoria del kernel, un proceso tiene que hacer una llamada al sistema,

\subsubsection{Stack - lectura y escritura}

Almacenamiento simple LIFO
Crece hacia abajo, desde el final del espacio de direcciones del proceso hasta el final de la BSS,

Cada funcion tiene su propio stack frame, que contiene las variables locales y los argumentos de la funcion,

Limitacion tipica entre 1 y 8MB, dependiendo de la arquitectura y el sistema operativo,

Utiliza el puntero de pila para reservar y liberar espacio en el stack (muy rapido)

\subsubsection{Heap - lectura y escritura}

Malloc, reserva momoria de forma dinamica
Crece hacia arriba, desde el final de la BSS hasta el final del espacio de direcciones del proceso

Utiliza el puntero de heap para reservar y liberar espacio en el heap (más lento que el stack)

\subsubsection{BSS y Data- lectura y escritura}

Variables estaticas y globales del programa,

\subsubsection{Text - lectura y ejecución}

Es cargado desde el ejecutable binario desde disco


\subsection{Procesos hijo e hilos}





